%\documentclass[12pt]{amsart}
\documentclass[12pt,a4paper]{article}
\usepackage{fullpage} 
\usepackage[normalem]{ulem}
\usepackage[T1]{fontenc}
\usepackage[utf8]{inputenc}
\usepackage[portuguese]{babel}
%\usepackage{emerald}
%\usepackage[cal=bickhams]{mathalpha}
%\usepackage{stmaryrd}

\setlength{\parskip}{.2em}
% \input{preamble.tex}
%%%%%%%%%%%%%%%%%%%%%%%%%% ``commonly used packages`` %%%%%%%%%%%%%%%%%%
\usepackage{amsfonts,amsmath,amsthm,amssymb}
%\usepackage{enumerate}
\usepackage{enumitem} 
%\usepackage{fullpage}
%\usepackage{appendix}
\usepackage{siunitx} % spacing in (big) numbers
%%%%%%%%%%%%%%%%%%% end of ``commonly used packages`` %%%%%%%%%%%%%%%%%%

\usepackage[hidelinks,linktoc=all]{hyperref}

%%%%%%%%%%%%%% \numberwithin requires the amsmath package
\numberwithin{table}{section}
\numberwithin{figure}{section}
\numberwithin{equation}{section}
%%%%%%%%%%%%%%%%%%%%%%%%%%%%%%%%% "for references" %%%%%%%%%%%%%%%%%%%%%%%%%%%%%%%%%
% to use bibtex, comment the following lines and adapt the end of the document
%%%%%%%%%
\usepackage[backend=biber,style=numeric,hyperref=true,isbn=false,doi=false]{biblatex}
\addbibresource{refs.bib}
%%%%%%%%%%%%%%%%%%%%%%%%%% end of "for references" %%%%%%%%%%%%%%%%%%%%%%%%%%%%
%%%%%%%%%%%%%%%%%%%%%%%%%%%%% "for drawings" %%%%%%%%%%%%%%%%%%%%%%%%%%%%
\usepackage{pgf} 
\usepackage{graphicx}
\usepackage{subcaption}
\usepackage{adjustbox}
%%%%%%%%%%%%%%%%%%%%%%%%%%%%% end of"for drawings" %%%%%%%%%%%%%%%%%%%%%
%%%%%%%%%%%%%%%%%%%%%%%%%%%%%%%%%% ``operators`` %%%%%%%%%%%%%%%%%%%
\DeclareMathOperator{\wilfoper}{W} %Wilf number
\DeclareMathOperator{\eliahouoper}{E} %Eliahou number
\DeclareMathOperator{\Frobeniusoper}{F} %Frobenius number
\DeclareMathOperator{\conductoroper}{c} %conductor
\DeclareMathOperator{\multiplicityoper}{m} %multiplicity
\DeclareMathOperator{\leftsoper}{L} %lefts
\DeclareMathOperator{\extleftsoper}{L^{\prime}} %extended lefts
\DeclareMathOperator{\primitivesoper}{P} %primitives
\DeclareMathOperator{\genusoper}{g} %genus
\DeclareMathOperator{\gapsoper}{G} %gaps
%\DeclareMathOperator{\maxprimoper}{a} %gaps

\DeclareMathOperator{\divisors}{\mathrm{Div}}
\DeclareMathOperator{\depth}{\mathrm{q}}
\DeclareMathOperator{\pdepth}{\mathrm{\pi}} %primitive depth

\DeclareMathOperator{\divides}{ | } %
\DeclareMathOperator{\dividess}{ |* } %
\DeclareMathOperator{\suchthat}{ : } %

%%%%%%%%%%%%%%%%%%%%%%%%%%%%% end of "operators" %%%%%%%%%%%%%%%%%%%%%%%%%%%%%

\newcommand{\titlefrobsetf}{\(\mathcal{N}_f\)}
\newcommand{\titlemaxprimcardn}{\(A_n\)}
\newcommand{\mainmap}{\(\Phi\)}

%%%%%%%%%%

\newcommand{\maxprimset}[1]{\mathcal{A}_{#1}}
\newcommand{\maxprimcard}[1]{{A}_{#1}}

\newcommand{\frobset}[1]{\mathcal{N}_{#1}}
\newcommand{\frobcard}[1]{{N}_{#1}}

\newcommand{\dfrobset}[1]{\mathcal{D}_{#1}}
\newcommand{\dfrobcard}[1]{{D}_{#1}}

%\newcommand{\divisors}{\mathrm{Div}}

%%%%%%%%%%%%%%%%%%%%%%%%%%%% ``theorems`` %%%%%%%%%%%%%%%%%%%%%%%%%
\newtheorem{theorem}{Theorem}[section] %(uncomment to number per section)
\newtheorem*{theorem*}{Theorem}
\newtheorem{lemma}[theorem]{Lemma}
\newtheorem{corollary}[theorem]{Corollary}
\newtheorem{proposition}[theorem]{Proposition}
\newtheorem*{proposition*}{Proposition}
%\newtheorem{algorithm}[theorem]{Algorithm}
\newtheorem{conjecture}[theorem]{Conjecture}
\newtheorem{fact}[theorem]{Fact}

\theoremstyle{definition}
\newtheorem{definition}[theorem]{Definition}
\newtheorem{notation}[theorem]{Notation}
\newtheorem{example}[theorem]{Example}

\theoremstyle{remark} 
%\newtheorem{example}[theorem]{Example}
\newtheorem{remark}[theorem]{Remark}
\newtheorem{question}[theorem]{Question}
\newtheorem{problem}[theorem]{Problem}

%%%%%%%%%%%%%%%%%%%%%%%%%%%% end of ``theorems`` %%%%%%%%%%%%%%%%%%%

\usepackage{colortbl}


\title{Contagem de semigrupos numéricos}
\author{Tema proposto por Manuel Delgado}
\date{2025/2026}

\begin{document}

\maketitle


\begin{abstract}
  
This note contains problems on numerical semigroups that are pottentialy adequate for undergraduate or young graduate students
  
\end{abstract}



%%%%%%%%%%% introduction %%%%%%%%%%
\section{Introdução}\label{sec:introducao}

Um semigrupo numérico \(S\) é um subconjunto de \(\mathbb{N}\) (o conjunto dos inteiros não negativos) tal que \(0 \in S\), \(S\) é estável para a adição e o seu complemento em \(\mathbb{N}\) é finito.

Uma boa introdução aos semigrupos numéricos é um livro de Rosales e García-Sánchez~\cite{RosalesGarcia2009Book-Numerical}. Lá encontramos a generalidade da notação e terminologia usada neste plano.

Ao longo deste plano de trabalhos, \(S\) representa um semigrupo numérico arbitrário. 

Um semigrupo numérico \(S\) tem um único conjunto de geradores que é minimal para a inclusão. Esse conjunto é finito e designamos os seus elementos por \emph{elementos primitivos}. Denotamos o conjunto dos elementos primitivos de~\(S\) por~\(\primitivesoper\). A sua cardinalidade, dita \emph{dimensão de imersão} de~\(S\) denota-se por~\(\lvert \primitivesoper\rvert\).

A \emph{multiplicidade} de~\(S\) é o mais pequeno inteiro positivo de~\(S\) e denota-se por~\(\multiplicityoper\).
O \emph{número de Frobenius} de~\(S\) é o maior inteiro que não pertence a~\(S\) e denota-se por~\(\Frobeniusoper\).
O \emph{condutor} de~\(S\) é \(\conductoroper=\Frobeniusoper +1\).
%
O conjunto dos \emph{elementos esquerdos} de~\(S\) consiste dos elementos de~\(S\) mais pequenos que~\(\Frobeniusoper\). Denota-se por \(\leftsoper\).

%%%%%%%%%%% Contar semigrupos numéricos %%%%%%%%%%
\section{Contar semigrupos numéricos}\label{sec:contar}
Aos invariantes combinatórios (género, número de Frobenius e primitivo máximo) estão associadas partições do conjunto de semigrupos numéricos e, consequentemente, sucessões numéricas  \(\left(n_g\right)\), …

Designamos genericamente o estudo destas sucessões por contagem de semigrupos numéricos, respetivamente por género, número de Frobenius e primitivo máximo.
Número de Frobenius – questao proposta por Wilf
Género - muito popular depois de algumas conjeturas propostas por Maria
Primitivo máximo - introduzida numa tese de doutoramento do CMUP, foi relevante nessa tese
A contagem por género despertou o interesse de muitos investigadores da área depois de Maria ter apresentado diversas conjunturas
Aquela com maior capacidade de atrair a atenção diz que a sucessão \(\left(n_g\right)\) tem (assintoticamente) um comportamento semelhante à sucessão de Fibonacci: \(\lim \frac{n_{g-1}+n_{g-2}}{n_g} = 1\)
Esta conjetura foi resolvida por Zhai, enquanto estudante. Outras conjeturas não tiveram a mesma sorte, o que me leva a pensar que os métodos usados não têm sido os mais adequados.



%%%%%%%%%%%% the numerical semigroups graph %%%%%%%%%%
\section{Grafo dos semigrupos numéricos e árvores geradoras}\label{sec:NSgraph}

Podemos construir um grafo de (todos os) semigrupos numéricos como segue: os vértices são os semigrupos numéricos; há uma aresta a ligar dois vértices se um deles puder ser obtido do outro por remoção de um gerador minimal 
Uma árvore geradora de um grafo é um subgrafo conexo, sem ciclos (diz-se uma árvore) que tem todos os vértices do grafo.
O subgrafo que se obtém considerando apenas as arestas que ligam semigrupos numéricos em que um se obtém do outro juntando-lhe o número de Frobenius (ou, o que é o mesmo, um se obtém do outro por remoção de um primitivo maior que o número de Frobenius) é uma árvore. É a (única) que aparece na literatura e costuma ser designada por árvore dos semigrupos numéricos. Vou designá-la por árvore clássica. Zhai obteve o resultado já referido recorrendo a uma análise muito detalhada da árvore clássica; vários investigadores têm usado a árvore clássica para obter os primeiros elementos da sucessão \(\left(n_g\right)\). O maior já obtido é \(n_{77} = \num{47008818196495180}\)


%%%%%%%%%%%%% Challenge %%%%%%%%%%%
\section{Desafio}\label{sec:desafio}

O desafio consiste em tentar considerar outras árvores geradoras, tendo em vista o estudo de assuntos relacionados com conjeturas para as quais a árvore clássica parece desadequada.

Referimos ainda a existência de software que ajudará na escolha das pistas a seguir na investigação, descartando outras. O pacote \texttt{numerigalsgps}~\cite{NumericalSgps1.4.0}, escrito na linguagem GAP~\cite{GAP4.14.0} e distribuído com o sistema computacional com o mesmo nome, destina-se a cálculos com semigrupos numéricos. Algoritmos desenvolvidos no decurso da investigação poderão ser implementados em GAP e adicionados ao pacote referido, o qual aceita contribuições.

Problema a ter em mente: a sucessão \(\left(n_g\right)\) é crescente?


%%%%%%%%%%%%%%%%%%%%%%%%%% references %%%%%%%%%%%%%%%%%%%%%%%%%%%%
%% uncomment to use biblatex  
\printbibliography
%%
%%uncomment to use bibtex
% \bibliographystyle{plainurl}
% %\bibliography{../../../../../bib/NumericalS-bib/numericals.bib,../../../../../bib/NumericalS-bib/preprints.bib,../../../../../bib/NumericalS-bib/software.bib,../../../../../bib/NumericalS-bib/slides.bib}
%%%%%%%%%
%\input{references}
\end{document}

